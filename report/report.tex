%% USEFUL LINKS:
%% -------------
%%
%% - UiO LaTeX guides:          https://www.mn.uio.no/ifi/tjenester/it/hjelp/latex/
%% - Mathematics:               https://en.wikibooks.org/wiki/LaTeX/Mathematics
%% - Physics:                   https://ctan.uib.no/macros/latex/contrib/physics/physics.pdf
%% - Basics of Tikz:            https://en.wikibooks.org/wiki/LaTeX/PGF/Tikz
%% - All the colors!            https://en.wikibooks.org/wiki/LaTeX/Colors
%% - How to make tables:        https://en.wikibooks.org/wiki/LaTeX/Tables
%% - Code listing styles:       https://en.wikibooks.org/wiki/LaTeX/Source_Code_Listings
%% - \includegraphics           https://en.wikibooks.org/wiki/LaTeX/Importing_Graphics
%% - Learn more about figures:  https://en.wikibooks.org/wiki/LaTeX/Floats,_Figures_and_Captions
%% - Automagic bibliography:    https://en.wikibooks.org/wiki/LaTeX/Bibliography_Management  (this one is kinda difficult the first time)
%%
%%                              (This document is of class "revtex4-1", the REVTeX Guide explains how the class works)
%%   REVTeX Guide:              http://www.physics.csbsju.edu/370/papers/Journal_Style_Manuals/auguide4-1.pdf
%%
%% COMPILING THE .pdf FILE IN THE LINUX IN THE TERMINAL
%% ----------------------------------------------------
%%
%% [terminal]$ pdflatex report_example.tex
%%
%% Run the command twice, always.
%%
%% When using references, footnotes, etc. you should run the following chain of commands:
%%
%% [terminal]$ pdflatex report_example.tex
%% [terminal]$ bibtex report_example
%% [terminal]$ pdflatex report_example.tex
%% [terminal]$ pdflatex report_example.tex
%%
%% This series of commands can of course be gathered into a single-line command:
%% [terminal]$ pdflatex report_example.tex && bibtex report_example.aux && pdflatex report_example.tex && pdflatex report_example.tex
%%
%% ----------------------------------------------------


\documentclass[english,notitlepage,reprint,nofootinbib]{revtex4-2}  % defines the basic parameters of the document
% For preview: skriv i terminal: latexmk -pdf -pvc filnavn
% If you want a single-column, remove "reprint"

% Allows special characters (including æøå)
\usepackage[utf8]{inputenc}
% \usepackage[english]{babel}

%% Note that you may need to download some of these packages manually, it depends on your setup.
%% I recommend downloading TeXMaker, because it includes a large library of the most common packages.

\usepackage{physics,amssymb}  % mathematical symbols (physics imports amsmath)
\usepackage{amsmath}
\usepackage{graphicx}         % include graphics such as plots
\usepackage{xcolor}           % set colors
\usepackage{hyperref}         % automagic cross-referencing
\usepackage{listings}         % display code
\usepackage{subfigure}        % imports a lot of cool and useful figure commands
% \usepackage{float}
%\usepackage[section]{placeins}
\usepackage{algorithm}
\usepackage[noend]{algpseudocode}
\usepackage{subfigure}
\usepackage{tikz}
\usetikzlibrary{quantikz}
% defines the color of hyperref objects
% Blending two colors:  blue!80!black  =  80% blue and 20% black
\hypersetup{ % this is just my personal choice, feel free to change things
	colorlinks,
	linkcolor={red!50!black},
	citecolor={blue!50!black},
	urlcolor={blue!80!black}}


% ===========================================


\begin{document}
	
	\title{Simulation of the 2D Ising model with Monte Carlo Markov Chain}  % self-explanatory
	\author{} % self-explanatory
	\date{\today}                             % self-explanatory
	\noaffiliation                            % ignore this, but keep it.
	
	%This is how we create an abstract section.
	\begin{abstract}
	We study the 2D ferromagnetic Ising Model with Monte Carlo Markov chain. We use the Metropolis algorithm to sample the probability distribution of the Ising model.
	As proven in 1944 by Lars Onsager$\colorbox{red}{Citation of his paper?}$, we have a phase transition at the critical temperature $T_c(L=\infty)=2.269 \mathrm{J / k}$. We find a matching numerical 
	critical temperature $T_c(L=\infty) = 2.29 \pm 0.04 \mathrm{J / k}$. To verify that the Metropolis algorithm is working correctly, we used the analytical solution in the case 
	of 2x2 lattice. Additionnaly we look at the distribution of our energy normalized by spin $\epsilon$ for a 20x20 lattice with different temperatures. We find that
	we temperature plays an important role in the shape of our distribution. Since we can only compute theoretical values when we reach equilibrium we used the "Burn-in" 
	method to determine the number of cycle needed to reach equilibrium. For a 20x20 lattice, we determine that around 10000 cycles is enough to reach equilibrium.
	  
	\end{abstract}
	\maketitle	
	
	
	% ===========================================
	\section{Introduction} \label{sec:introduction}
	%
	The Ising model is a statistical mechanics model that describes the ferromagnetism by a system of spins on a lattice. 
	Ferromagnetism is a property of certain materials allowing it to form a permanent magnet for example. One important concept in the Ising model for dimension 
	higher than 1 is phase transition. A phase transition is an extreme change in the physical properties. In our case, the model will go high magnetization to close to 
	0 with increasing temperature. $\colorbox{red}{More detail about Ising model? Other use in science perhaps}$\\
	
	There is no analytical solution for the Ising model in dimensions $>$ 2. However, we can use the Monte Carlo Markov Chain (MCMC) algorithm to simulate our model. 
	MCMC was developed in the 40s by Stanislaw Ulam and Jon Von Neumann $\colorbox{red}{Citation of his paper?}$. The basic idea is to use randomness to solve deterministic problem and as shown great 
	results in many situations $\colorbox{red}{some example?}$ in computational physics. In this
	paper we will apply the Monte Carlo methods to solve the 2D Ising Model and we will use the 
	Metropolis algorithm to sample the distribution. As explained in more details in section
	\ref{sec:theory} we will use the relation between finite lattice and infinite lattice to 
	estimate the critical temperature, $T_c(L=\infty)$ solved analytically by Lars Onsager in 1944. \\
	
	In section \ref{sec:theory}, we will introduce a formal explanation of the Ising Model 
	with the analytical solution for the 2x2 case used as a benchmark for our algorithm. In the
	next section \ref{sec:methods} we will discuss the numerical methods and the algorithms 
	used. Section \ref{sec:results} and section \ref{sec:discussion} respectively presents our
	results from our simulation and an analysis of them. Finally section \ref{sec:conclusion} summarize the main points of this paper.   

	\section{Theory}\label{sec:theory}
	\subsection{2D Ising Model} \label{subsec:Ising}
	The 2D Ising model is a 2 dimensional lattice of equal length and row (L) compose of spins
	$s_i$. So an lattice of size L will wield $\rm LxL=N$ spins. Each spin can take one of the
	following values $\{-1,	1\}$ representing their magnetic dipole. Every spin can interact 
	with their next neighbor meaning that every spin can have up to four interactions. One 
	important detail in physics is what to do with the boundaries. In our case we will consider 
	a periodic boundary condition. This means that we will always consider four interactions
	for each spin. This is the geometrical equivalent of having a lattice in the shape of a 
	torus. We will denote the spin configuration by $\textbf{s}=\{s_1, s_2, ..., s_N\}$. This allows us 
	to write the Hamiltonian 
	
	\begin{equation}
		H(\textbf{s}) = -J \sum_{<kl>} s_ks_l + h \sum_{i=1}^{N} s_i \label{eq:hamilton}
	\end{equation}
	
	Eq. \ref{eq:hamilton} is composed of two terms. The first one is the contribution of every 
	neighbors. $<kl>$ represents the neighboring pairs of k and l. $J$ is the coupling constant 
	and represent the strength of spin interaction. In this paper every results have been 
	produce with $J=1$. The second term is here to account for a potential external magnetic 
	field with $h$ acting as the strength of this magnetic field. We will not consider an 
	external magnetic field in this paper so $h=0$ and eq. \ref{eq:hamilton} becomes
	
	\begin{equation}
		H(\textbf{s}) = -J \sum_{<kl>} s_ks_l \label{eq:Hamiltonian}
	\end{equation}
	
	In statistical mechanics we use probability and observables. In our problem we have two 
	observables the magnetization and the energy respectively 
	
	\begin{align}
		M(\textbf{s}) &= \sum_{i=1} s_i \label{eq:M}\\
		E(\textbf{s}) &= H(\textbf{s}) = -J \sum_{<kl>} s_ks_l \label{eq:E}
	\end{align}

	$M$ is the magnetization summing all spins in the lattice. Since we will study these 
	observable for different lattice size we introduce the normalized by spin magnetization and
	energy 
	
	\begin{align}
		m(\textbf{s}) &= \frac{M(\textbf{s})}{N} \label{eq:m} \\   
		\epsilon(\textbf{s}) &= \frac{E(\textbf{s})}{N}  \label{eq:epsilon}
	\end{align}

	
	Given a temperature $T$ to determine the probability of being in the configuration
	$\textbf{s}$ we will use the Boltzmann distribution with the partition function $Z$
	
	\begin{align}
		p(\textbf{s},T) &= \frac{1}{Z} e^{-\beta E(\textbf{s})} \label{eq:B_probability} \\
		Z &= \sum_{\text{\textbf{s}}} e^{-\beta E(\textbf{s})} \label{eq:partition_fun}
	\end{align}

	where $\beta = \frac{1}{kT}$, $k$ the Boltzmann constant. the summation for the partition
	function means that we iterate over every possible s\textbf{s}. This numbers scale as $2^N$
	showing with analytical calculation of the partition for big lattice becomes quickly 
	impossible. \\
	
	In statistical mechanics we can only experimentally calculate the averaged values of our 
	observable. Meaning that for a discrete observable $A(X)$ with probability $p(X)$ we have 
	\begin{equation}
		<A> = \sum_{X}A(X)p(X) \label{eq:expectation value}
	\end{equation} 
	
	We can then apply eq. \ref{eq:expectation value} to our observables. 
 	
	\subsection{2x2 analytical solution}
	
	Since it gets quickly impossible to calculate the probability distribution due to the 
	partition function. We will first solve the 2x2 case where we only have $2^4=16$ possible
	$\textbf{s}$. 
	\begin{table}[h!]
	\centering
	\begin{tabular}{|c|c|c|c|}
		\hline
		\# $s_i=1$ & E & M & degeneracy \\
		\hline
		\hline
		0 & -4 &  -4 &  1  \\
		1 & 0&   -2&   4  \\ 
		2 & 0 & 0&  4  \\
		2 & 4 & 0 &  2 \\
		3 & 0  & 2 & 4 \\
		4 & -4 & 4 & 1 \\
		\hline
	\end{tabular}
	\caption{ Summary of all possible energy E, magnetization M and the degeneracy is the 
	number of spin configuration with the same characteristic} \label{tab:summary2x2}
	\end{table} 
	
	Table \ref{tab:summary2x2} shows all possibility in the 2x2 case and we see that summing all
	of the degeneracy gives us 16 assuring us that we covered every single possibility. Now that
	we have found these values. We can proceed in the calculation of the partition function
	
	\begin{equation}
		Z = 12 + 4\cosh(8\beta) \label{eq:Analytical_partition} 
	\end{equation}

	
	
	% ===========================================
	\section{Methods}\label{sec:methods}
	%
	We will use

	
	% ===========================================
	\subsection*{The algorithm}\label{sec:algorithm}
	%

	% ===========================================
	\section{Results}\label{sec:results}
	%

	
	
	% ===========================================
	\section{Discussion}\label{sec:discussion}
	%

	
	% ===========================================
	\section{Conclusion}\label{sec:conclusion}

	\onecolumngrid
	
	%\bibliographystyle{apalike}
	\bibliography{ref}
	
	
\end{document}
