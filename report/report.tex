%% USEFUL LINKS:
%% -------------
%%
%% - UiO LaTeX guides:          https://www.mn.uio.no/ifi/tjenester/it/hjelp/latex/
%% - Mathematics:               https://en.wikibooks.org/wiki/LaTeX/Mathematics
%% - Physics:                   https://ctan.uib.no/macros/latex/contrib/physics/physics.pdf
%% - Basics of Tikz:            https://en.wikibooks.org/wiki/LaTeX/PGF/Tikz
%% - All the colors!            https://en.wikibooks.org/wiki/LaTeX/Colors
%% - How to make tables:        https://en.wikibooks.org/wiki/LaTeX/Tables
%% - Code listing styles:       https://en.wikibooks.org/wiki/LaTeX/Source_Code_Listings
%% - \includegraphics           https://en.wikibooks.org/wiki/LaTeX/Importing_Graphics
%% - Learn more about figures:  https://en.wikibooks.org/wiki/LaTeX/Floats,_Figures_and_Captions
%% - Automagic bibliography:    https://en.wikibooks.org/wiki/LaTeX/Bibliography_Management  (this one is kinda difficult the first time)
%%
%%                              (This document is of class "revtex4-1", the REVTeX Guide explains how the class works)
%%   REVTeX Guide:              http://www.physics.csbsju.edu/370/papers/Journal_Style_Manuals/auguide4-1.pdf
%%
%% COMPILING THE .pdf FILE IN THE LINUX IN THE TERMINAL
%% ----------------------------------------------------
%%
%% [terminal]$ pdflatex report_example.tex
%%
%% Run the command twice, always.
%%
%% When using references, footnotes, etc. you should run the following chain of commands:
%%
%% [terminal]$ pdflatex report_example.tex
%% [terminal]$ bibtex report_example
%% [terminal]$ pdflatex report_example.tex
%% [terminal]$ pdflatex report_example.tex
%%
%% This series of commands can of course be gathered into a single-line command:
%% [terminal]$ pdflatex report_example.tex && bibtex report_example.aux && pdflatex report_example.tex && pdflatex report_example.tex
%%
%% ----------------------------------------------------


\documentclass[english,notitlepage,reprint,nofootinbib]{revtex4-2}  % defines the basic parameters of the document
% For preview: skriv i terminal: latexmk -pdf -pvc filnavn
% If you want a single-column, remove "reprint"

% Allows special characters (including æøå)
\usepackage[utf8]{inputenc}
% \usepackage[english]{babel}

%% Note that you may need to download some of these packages manually, it depends on your setup.
%% I recommend downloading TeXMaker, because it includes a large library of the most common packages.

\usepackage{physics,amssymb}  % mathematical symbols (physics imports amsmath)
\include{amsmath}
\usepackage{graphicx}         % include graphics such as plots
\usepackage{xcolor}           % set colors
\usepackage{hyperref}         % automagic cross-referencing
\usepackage{listings}         % display code
\usepackage{subfigure}        % imports a lot of cool and useful figure commands
% \usepackage{float}
%\usepackage[section]{placeins}
\usepackage{algorithm}
\usepackage[noend]{algpseudocode}
\usepackage{subfigure}
\usepackage{tikz}
\usetikzlibrary{quantikz}
% defines the color of hyperref objects
% Blending two colors:  blue!80!black  =  80% blue and 20% black
\hypersetup{ % this is just my personal choice, feel free to change things
	colorlinks,
	linkcolor={red!50!black},
	citecolor={blue!50!black},
	urlcolor={blue!80!black}}


% ===========================================


\begin{document}
	
	\title{Looking at the Ising problem with Monte Carlo Markov chain}  % self-explanatory
	\author{} % self-explanatory
	\date{\today}                             % self-explanatory
	\noaffiliation                            % ignore this, but keep it.
	
	%This is how we create an abstract section.
	\begin{abstract}
	We study the 2D ferromagnetic Ising Model with Monte Carlo Markov chain. We use the Metropolis algorithm to sample the probability distribution of the Ising model.
	As proven in 1944 by Lars Onsager, we have a phase transition at the critical temperature $T_c(L=\infty)=2.269 \mathrm{J / k}$. We find a matching numerical 
	critical temperature $T_c(L=\infty) = 2.29 \pm 0.04$. To verify that the Metropolis algorithm is working correctly, we used the analytical solution in the case 
	of 2x2 lattice. Additionnaly we look at the distribution of our energy normalized by spin $\epsilon$ for a 20x20 lattice with different temperatures. We find that
	we temperature plays an important role in the shape of our distribution. Since we can only compute theoretical values when we reach equilibrium we used the "Burn-in" 
	method to determine the number of cycle needed to reach equilibrium. For a 20x20 lattice, we determine that around 10000 cycles is enough to reach equilibrium.
	  
	\end{abstract}
	\maketitle	
	
	
	% ===========================================
	\section{Introduction} \label{sec:introduction}
	%
	The Ising model is a statistical mechanics model that describes the ferromagnetism by a system of spins on a lattice. 
	Ferromagnetism is a property of certain materials allowing it to form a permanent magnet for exemple. One important concept in the Ising model for dimension 
	higher than 1 is phase transition. A phase transition is an extreme change in the physical properties. In our case, the model will go high magnetization to close to 
	0 with increasing temperature. 

	\section{Theory}\label{sec:theory}
	% ===========================================
	\section{Methods}\label{sec:methods}
	%
	We will use

	
	% ===========================================
	\subsection*{The algorithm}\label{sec:algorithm}
	%

	% ===========================================
	\section{Results}\label{sec:results}
	%

	
	
	% ===========================================
	\section{Discussion}\label{sec:discussion}
	%

	
	% ===========================================
	\section{Conclusion}\label{sec:conclusion}

	\onecolumngrid
	
	%\bibliographystyle{apalike}
	\bibliography{ref}
	
	
\end{document}
